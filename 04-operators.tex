\documentclass[myc.tex]{subfiles}

\begin{document}
\chapter{Выражения и операторы}
\label{ch:operators}

\footnote{Существует путаница в терминологии. Под словом <<оператор>> может пониматься как operator так и statement. Первое является неправильным переводом (правильно -- <<операция>>), а statement правильнее переводить как <<инструкция>>.

В этом учебнике будет использоваться калька с английского языка <<оператор>> в значении слова operator.}


\section{Выражения}

\section{Операторы присваивания}
=




%+=
%*= //?

\section{Инкремент и декремент}





\section{Арифметические операторы}



\section{Операторы сравнения}


\section{Логические операторы}


\section{Битовые операции}
%\subsection{Операторы битового сдвига}

+указатели

sizeof()(

\section{Приведение типов}

\section{Приоритеты операторов}




\subsection*{Контрольные вопросы}
\begin{enumerate}
\item 55
\end{enumerate}
\end{document}
