\documentclass[myc.tex]{subfiles}

\begin{document}
\chapter{000 типы данных 2}
\section{\texttt{enum}. Перечисления}
\index{enum}
+синтаксическая диаграмма как в if
\section{\texttt{union}}
\index{union}
\section{\texttt{struct}. Структуры}
\index{struct}
\section{Массивы}
\section{Указатели}

%typedef void (*Callback)(void *arg, int fd, uint32_t events);







todo

одна и та же область памяти может 000поразному см(приведение типов и union







\chapter{\texttt{printf} и \texttt{scanf}}
\subsection*{Контрольные вопросы}
\begin{enumerate}
\item 55
\end{enumerate}














\chapter{Стандартная библиотека}

99

<assert.h>
<complex.h>
<ctype.h>
<errno.h>
<fenv.h>
<float.h>
<inttypes.h>
<iso646.h>
<limits.h>
<locale.h>
<math.h>
<setjmp.h>
<signal.h>
<stdarg.h>
<stdbool.h>
<stddef.h>
<stdint.h>
<stdio.h>
<stdlib.h>
<string.h>
<tgmath.h>
<time.h>
<wchar.h>
<wctype.h

\section{<assert.h>}
Пример использования приведён в %TODO

проверяет существование макроса NDEBUG.



%\section{complex.h}
%Тригонометрические функции:
%\vspace{4pt}
%\noindent
%\begin{small}
%\begin{tabularx}{\textwidth}{|X|}
%\hline
%\textbf{cacos} -- косинус\\
%\hline
%\texttt{double complex cacos(double complex z);}\newline
%\texttt{float complex cacosf(float complex z);}\newline
%\texttt{long double complex cacosl(long double complex z);}\\
%\hline
%The cacos functions compute the complex arc cosine of z, with branch cuts outside the
%interval %[−1, +1] along the real axis.
%Returns
%3
%The cacos functions return the complex arc cosine value, in the range of a strip
%mathematically unbounded along the imaginary axis and in the interval [0, %π ] along the
%real axis.
%\\
%\hline
%\end{tabularx}
%\end{small}
%\vspace{4pt}
%\noindent
%\begin{small}
%\begin{tabularx}{\textwidth}{|X|}
%\hline
%\textbf{casin} -- \\
%\hline
%\texttt{double complex casin(double complex z);
%float complex casinf(float complex z);Мlong double complex casinl(long double complex z);}\\
%\hline
%\\
%\hline
%\end{tabularx}
%\end{small}








\section{ctype.h}

для работы с символом?!

\noindent
\begin{small}
\begin{tabularx}{\textwidth}{|l|X|}
\hline
\texttt{int isalnum(int c)} & The isalnum function tests for any character for which isalpha or isdigit is true.\\
\hline
\texttt{int isalpha(int c)} & The isalpha function tests for any character for which isupper or islower is true,
or any character that is one of a locale-specific set of alphabetic characters for which none of iscntrl, isdigit, ispunct, or isspace is true. 170) In the "C" locale,
isalpha returns true only for the characters for which isupper or islower is true.\\
\hline
\texttt{int isblank(int c)} & \\
\hline
\texttt{} & \\
\hline
\texttt{} & \\
\hline
\texttt{} & \\
\hline
\texttt{} & \\
\hline
\texttt{} & \\
\hline
\texttt{} & \\
\hline
\end{tabularx}
\end{small}


\end{document}
