\documentclass[myc.tex]{subfiles}

\begin{document}
\chapter{Инструкции (statements)}

\footnote{Будет использоваться термин <<исрукция>>


Существует путаница в терминологии, как переводить слово <<statement>>. См. примечание на странице~\pageref{ch:operators} в главе~\ref{ch:operators}~<<\nameref{ch:operators}>>.

Для создания большей путаницы в этой главе будет использоваться оба варианта перевода.}


\section*{Как создавать и читать блок-схемы}
\addcontentsline{toc}{section}{Как создавать и читать блок-схемы}
\index{блок-схема}



\section{Выражения как инструкции}


+инициализация переменных сразу.




\section{\texttt{if}. Инструкция ветвления}
\index{if}

\noindent
\includegraphics[scale=0.077]{images-gen/if.png}
\begin{lstlisting}
if ($\textit{выражение}$)
    statement
    
if ($\textit{выражение}$)
    statement
else
    statement
\end{lstlisting}






dfg dfg d gdfg dfg dfg d gdfg dfg dfg d gdfg dfg dfg d gdfg dfg dfg d gdfg dfg dfg d gdfg dfg dfg d gdfg dfg dfg d gdfg dfg dfg d gdfg dfg dfg d gdfg dfg dfg d gdfg dfg dfg d gdfg dfg dfg d gdfg dfg dfg d gdfg dfg dfg d gdfg dfg dfg d gfg dfg dfg d g
5

\begin{lstlisting}[caption=Пример использования оператора if]
if (a < 5) {
    b = 0;
}
\end{lstlisting}

Рекомендуется всегда ставить скобки.


Поясним на примере. Предположим у нас был следующий код:
\begin{lstlisting}
if (a < 5)
    b = 0;
\end{lstlisting}

И возникла потребность добавить ещё одну строку:
\begin{lstlisting}
if (a < 5)
    b = 0;
    c = 0; //error
\end{lstlisting}

Судя по отступу в 3 строке, программист хотел, чтобы переменной \texttt{c} присваивался ноль, если условие в 1 строке истинно. Однако, программа работает не так, как предполагал программист. Вне зависимости от истинности выражения, присваивание происходит всегда, так как записано за пределами конструкции \texttt{if}. Правильный вариант кода:
\begin{lstlisting}
if (a < 5) {
    b = 0;
    c = 0;
}
\end{lstlisting}




В главе bool было сказано, что типа таого нета

\begin{lstlisting}
if (0)   { puts("0"); }
if (1)   { puts("1"); }
if (42)  { puts("42"); }
if (-1)  { puts("-1"); }
if (-42) { puts("-42"); }
\end{lstlisting}

\begin{verbatim}
0
45456
545
\end{verbatim} 


\section{\texttt{switch}}
\index{switch}

\section{\texttt{while}. Цикл}
\index{while}

\section{\texttt{do}}
\index{do}

\section{\texttt{for}}
\index{for}



c99 for(int i;;i++)

c89 for(;;i++)


goto\index{goto}

break\index{break}

итд








\subsection*{Контрольные вопросы}
\begin{enumerate}
\item 55
\end{enumerate}
\end{document}
