\documentclass[myc.tex]{subfiles}

\begin{document}
\chapter*{Предисловие}
\addcontentsline{toc}{chapter}{Предисловие}

Язык программирования Си является языком общего назначения и был создан 1972--1973 годах сотрудником Bell Labs Деннисом Ритчи (Dennis Ritchie) как развитие языка Би. Первоначально был разработан для реализации операционной системы UNIX, Однако, удачное сочетание лаконичности конструкций и богатства выразительных возможностей позволило языку Си быстро распространиться и стать наиболее популярным языком прикладного и системного программирования.

%TODO драйвера/ядра/модули ядер/nginx/прочее

Язык Си впервые был стандартизован в 1989 году и получил название \texttt{ANSI C}.

В настоящее время (\the\year{ }год) существуют несколько стандартов языка Си:
\begin{itemize}
\item \texttt{ANSI C} (\texttt{C89})
\item \texttt{ISO C} (\texttt{C99})
\item \texttt{C11}
\item \texttt{C18}
\end{itemize}

Все стандарты по большей части обратно совместимы. Примеры кода в этом учебнике написаны для стандарта \texttt{C99}, но с большой вероятностью будут успешно компилироваться и работать и в более новых компиляторах, в том числе и языка C++ (читается как <<си-плюс-плюс>>).





\subsection*{Структура книги //todo}

Учебник разбит на уроки с постепенно возрастающей сложностью, которые следует изучать по порядку.

блоки кода

\lstinputlisting[caption=Исходный код программы <<Hello World>>]{examples/2-hello.c}

Результат выполнения программы:
\begin{verbatim}
Hello World!
\end{verbatim} 


Основные термины будут выделены \textit{курсивом}. В тексте используются англоязычные термины или их устоявшаяся калька с английского языка (например, operator -- <<оператор>>, а не <<операция>>). По возможности в скобках будут приведены их русскоязычные аналоги, однако читателю стоит иметь в виду, одни и те же термины могут называться по разному в разных источниках.

После каждого урока расположены контрольные вопросы.




\subsection*{Какие темы не рассматриваются //todo}
В этом учебнике не будут рассматриваться следующие темы:
\begin{itemize}
\item настройка интегрированной среды разработки (IDE);
\item способы и методы отладки приложений.
\end{itemize}

Предполагается, что у читателя есть установленный и настроенный компилятор языка Си.


\subsection*{Благодарности //todo}
000

* разработчикам SQLite за инструмент для генерации синтаксических диаграмм.







\subsection*{Прочее //todo}
Исходный код учебника доступен по адресу \url{https://github.com/ermakov0/beginner-book-c}. Присылайте свои замечания (в том числе об опечатках или замеченных неточностях) в issues, все они будут внимательно рассмотрены и проигнорированы.
\end{document}
