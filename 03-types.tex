\documentclass[myc.tex]{subfiles}

\begin{document}
\chapter{Примитивные типы данных}

\section{\texttt{int}. Целые числа}

\begin{figure}[th]
\centering
\includegraphics[scale=\varscale]{images-gen/type-int.png}
\caption{Упрощенная синтаксическая диаграмма с наиболее используемыми целочисленными типами данных}
\label{fig:type-int}
\end{figure}

Язык Си имеет пять различных типов для хранения целых чисел:

\begin{itemize}
\item \texttt{char}\index{char}: один байт. Используется для хранения одного символа\footnote{в ASCII или однобайтовой кодировке};
\item \texttt{short}\index{short}: 16 бит в большинстве систем. Также известен как \texttt{short int}. Используется редко;
\item \texttt{int}\index{int}: 
\item \texttt{long}\index{long}: Также известен как \texttt{long int};
\item \texttt{long long}\index{long long}: Также известен как \texttt{long long int}.
\end{itemize}

В зависимости от архитектуры процессора, разрядности компилятора (?) размеры могут отличаться. Например, при компиляции кода для архитектуры AVR (8-ми битные микроконтроллеры, в том числе и платы arduino) размер типа данных \texttt{int} будет составлять 2 байта.

Следующий код, собранный под архитектуры x86 и amd64, будет выводить различный результат:%\footnote{в примере использован формат вывода \texttt{\%lu}, вместо привычного \texttt{\%d}, потому что оператор \texttt{sizeof} возвращает число с типом данных \texttt{unsigned long}. Если попытаться использовать \texttt{\%d}, то при выводе произойдёт неявная конвертация беззнакового числа в знаковое и компилятор выдаст предупреждение:
%\texttt{warning: format specifies type 'int' but the argument has type 'unsigned long' [-Wformat]}
%}:
\begin{lstlisting}
printf("%lu\n", sizeof(char));
printf("%lu\n", sizeof(short));
printf("%lu\n", sizeof(int));
printf("%lu\n", sizeof(long));
printf("%lu\n", sizeof(long long));
\end{lstlisting}
Результат выполнения программы:
\begin{verbatim}
1
2
4
8
8
\end{verbatim} 


% char - стандарт гарантирует мнимальный (?) размер в один байт

Примеры целых чисел: \texttt{-13 0 42}.





Беззнаковые: \texttt{unsigned}\index{unsigned}.

\begin{small}
\begin{center}
\begin{tabular}{|c|c|c|}
\hline
 & \textbf{unsigned} & \textbf{signed}\\
\hline
00000000 & 0 & 0\\
00000001 & 1 & 1\\
00000010 & 2 & 2\\
... & ... & ...\\
01111111 & 127 & 127\\
\textbf{1}0000000 & 128 & 0\\
\textbf{1}0000001 & 129 & -1\\
... & ... & ...\\
\textbf{1}1111101 & 253 & -125\\
\textbf{1}1111110 & 254 & -126\\
\textbf{1}1111111 & 255 & -127\\%TODO check!
\hline
\end{tabular}
\end{center}
\end{small}

Типичные размеры для 32 битной архитектуры.

\noindent
\begin{small}
\begin{tabularx}{\textwidth}{|l|l|X|}
\hline
\textbf{Тип} & \textbf{Число бит} & \textbf{Диапазон возможных\newline значений}\\
\hline
\texttt{unsigned char} & 8 & от 0 до 255 ($2^8-1$)\\ 
\texttt{char} & 8 & от -128 до 127 ($2^7-1$)\\ % 128!
\hline
cell7 & 16 & cell9 \\ 
cell7 & 32 & cell9 \\ 
cell7 & 32 & cell9 \\ 
cell7 & 64 & cell9 \\ 
\hline
\end{tabularx}
\end{small}


Переполнение знакового и беззнакового целого числа

//UB!

таблички

11111110 254
11111111 255
00000000
00000001




\subsection{\texttt{size\_t}}
Для различных счётчиков (от 0) рекомендуется использовать специальный тип данных \texttt{size\_t}:
\begin{lstlisting}
#include <stddef.h>

size_t i, j;
\end{lstlisting}



\subsection{\texttt{int*\_t} и \texttt{uint*\_t}. Типы с заданным размером}
Если потребуется хранить числа заданного размера, то заголовочный файл \texttt{stdint.h} объявляет ещё несколько целочисленных типов:
\texttt{int8\_t}, \texttt{int16\_t}, \texttt{int32\_t}, \texttt{int64\_t},
\texttt{uint8\_t}, \texttt{uint16\_t}, \texttt{uint32\_t}, \texttt{uint64\_t}.

\begin{lstlisting}
#include <stdint.h>

// 1 byte -> [0-255] or [0x00-0xFF]
uint8_t number8;

// 2 bytes -> [0-65535] or [0x0000-0xFFFF]
uint16_t number16;

// 4 bytes -> [0-4294967295] or [0x00000000-0xFFFFFFFF]
uint32_t number32;

// 8 bytes -> [0-18446744073709551615] or [0x0000000000000000-0xFFFFFFFFFFFFFFFF]
uint64_t number64;
\end{lstlisting}



































\section[\texttt{float} и \texttt{double}. Числа с плавающей точкой]{\texttt{float} и \texttt{double}. Числа\\ с плавающей точкой}


float

double

long double\index{long double}

































\section{\texttt{bool}. Логический тип данных}

Условное выражение может принимать одно из двух значений: \textit{истина} (\texttt{true}) или \textit{ложь} (\texttt{false}). Многие операторы языка Си возвращают логический тип данных, например, операторы сравнения:

\begin{lstlisting}
4 == 4  //true
5 > 4   //true
5 < 0   //false
\end{lstlisting}

Логические выражения могут быть \textit{истинными} и \textit{ложными} (когда результат вычисления выражения равен \texttt{true} или \texttt{false} соответственно).

Изначально в стандарте C89 логического типа не существовало, и значение \texttt{true} представлено любым ненулевым значением, включая отрицательные числа. Значение \texttt{false} всегда представлено нулем. 

Стандарт C99 оставляет совместимость с предыдущим стандартом и вводит новый тип данных: \texttt{\_Bool}. Для удобства использования логического типа данных и совместимостью с C++ стандартная библиотека предоставляет заголовочный файл \texttt{stdbool.h} с макросами \texttt{bool}, \texttt{true} и \texttt{false}:

\begin{lstlisting}
#include <stdbool.h>

bool is_odd = true;
bool is_fail = false;
\end{lstlisting}

Рекомендуется давать имена переменных логического типа с префиксом \texttt{is\_} для повышения уровня читабельности кода.




































%\section{\texttt{complex}. Комплексные числа}
%#include <complex.h>
%#include <stdio.h>
%void example (void)
%{
%complex double z = 1.0 + 3.0*I;
%printf ("Phase is %f, modulus is %f\n", carg (z), cabs (z));
%}

\vfill
\subsection*{Контрольные вопросы}
\begin{enumerate}
\item 
\end{enumerate}
\end{document}
