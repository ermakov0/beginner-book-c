\documentclass[myc.tex]{subfiles}

\begin{document}
\chapter[Лексические элементы языка Си]{Лексические элементы\\ языка Си}
В этой главе описываются лексические элементы, из которых состоит исходный код на языке Си после его обработки препроцессором. Такие элементы называются \textit{токенами}\index{токен}\footnote{так же известны как \textit{лексемы}.}. Они являются минимальной единицей языка.

Существуют пять типов токенов:
\begin{enumerate}
\item Идентификаторы;
\item Ключевые слова;
\item Константы;
\item Операторы;
\item Разделители.
\end{enumerate}

Пробельные символы (пробел, табуляция, перевод строки) тоже относят к разделителям, поскольку в исходном коде они разделяют элементы на отдельные токены.
















\section*{Как читать синтаксические\\ диаграммы}
\addcontentsline{toc}{section}{Как читать синтаксические диаграммы}
\index{синтаксическая диаграмма}

\textit{Синтаксис}\index{синтаксис} языка программирования -- набор правил, описывающий последовательность токенов, считающиеся валидной программой или её фрагментом.

Синтаксические диаграммы -- графическое представление синтаксиса языка программирования. Они однозначно отражают, как компилятор будет разбирать исходный код и осуществлять проверку синтаксиса.

Синтаксические диаграммы читаются слева направо и вдоль стрелок. Ключевые слова, операторы и разделители отображаются внутри закруглённых прямоугольников. В прямоугольниках заключены другие семантические диаграммы либо токены, требующие дальнейшего разбора.

Пример упрощенной синтаксической диаграммы:

\begin{center}
\includegraphics[scale=\varscale]{images-gen/example-if.png}
\end{center}

Если синтаксическая диаграмма разветвляется, то при чтении можно выбрать любой путь.

\begin{center}
\includegraphics[scale=\varscale]{images-gen/example-digit.png}
\end{center}




















\section[Идентификаторы]{Идентификаторы \hfill\includegraphics[height=20pt]{images-gen/id.png}}

\textit{Идентификаторы}\index{идентификатор} используются для именования переменных, функций, новых типов данных и макросов препроцессора. Могут состоять из латинских букв, цифр и знака нижнего подчёркивания \texttt{\_}.

Первый символ идентификатора не может быть цифрой. Максимальная длина идентификатора в языке Си~-- 31~символ.

Идентификаторы являются чувствительными к регистру, строчные и прописные буквы различаются, поэтому \texttt{foo} и \texttt{FOO} -- это два разных идентификатора.

Примеры идентификаторов: \texttt{foo} \texttt{bar} \texttt{\_var} \texttt{sum\_all} \texttt{arr2} \texttt{firstName} \texttt{homeworkScore} \texttt{AbstractCar} \texttt{SIZE\_ARRAY}.

Называть переменные и функции рекомендуется, используя \texttt{верблюжийРегистр}, константы -- в \texttt{ВЕРХНЕМ\_РЕГИСТРЕ}.

\begin{lstlisting}
#define SIZE_ARRAY 12

int counter;
float result;

void myFunc() {}
\end{lstlisting}
\noindent
\begin{small}
\begin{tabularx}{\textwidth}{|l|X|}
\hline
\textbf{Строки} & \textbf{Описание}\\
\hline
1 & идентификатор \texttt{SIZE\_ARRAY} используется для\newline объявления макроса препроцессора\\ % *не* константы!
\hline
3-4 & \texttt{counter} и \texttt{result} -- имена переменных\\
\hline
6 & \texttt{myFunc} -- имя функции\\
\hline
\end{tabularx}
\end{small}


























\section[Ключевые слова]{Ключевые слова \hfill\includegraphics[height=20pt]{images-gen/int.png}}

\textit{Ключевые слова}\index{ключевое слово} -- это специальные идентификаторы, зарезервированные для для использования, и являются часть грамматики языка. Ключевые слова невозможно использовать для других целей.

Ключевые слова, описанные в стандарте ANSI C89:

\noindent
\texttt{auto} \texttt{break} \texttt{case} \texttt{char} \texttt{const} \texttt{continue} \texttt{default} \texttt{do} \texttt{double} \texttt{else} \texttt{enum} \texttt{extern} \texttt{float} \texttt{for} \texttt{goto} \texttt{if} \texttt{int} \texttt{long} \texttt{register} \texttt{return} \texttt{short} \texttt{signed} \texttt{sizeof} \texttt{static} \texttt{struct} \texttt{switch} \texttt{typedef} \texttt{union} \texttt{unsigned} \texttt{void} \texttt{volatile} \texttt{while}.

Стандарт C99 дополняет список ключевых слов:

\noindent
\texttt{inline} \texttt{\_Bool} \texttt{\_Complex} \texttt{\_Imaginary}.

Различные дополнения компиляторов вводят дополнительные ключевые слова начинающиеся с двух символов подчёркивания (\texttt{\_\_}), поэтому нужно с осторожностью использовать идентификаторы, начинающиеся с таких символов.






















\section{Константы}

% не путать с макросами проепроцессора!

\textit{Константа}\index{константа} -- это числовое или буквенное значение. Константы имеют конкретный тип данных, однако, при желании можно использовать приведение типов, если поведение по умолчанию компилятора не устраивает. 

\subsection[Целочисленные константы]{Целочисленные константы \hfill\includegraphics[height=20pt]{images-gen/const-int.png}}

Целочисленная константа -- последовательность цифр с необязательным префиксом для указания системы исчисления.

Префикс \texttt{0x} или \texttt{0X} указывает, что число записано в шестнадцатеричной системе исчисления. Для чисел в шестнадцатеричной системе исчисления помимо цифр \texttt{0-9} используются латинские буквы от \texttt{a} до \texttt{f} или от \texttt{A} до \texttt{F}.

Префикс \texttt{0} -- в восьмеричной. Для восьмеричной системы исчисления используются только цифры от \texttt{0} до \texttt{7}.

Префикс \texttt{0b} или \texttt{0B} для двоичной системы исчисления не входит в стандарт языка Си, и поддерживается не всеми компиляторами.
%You can use binary literals. They are standardized in C++14. For example,
%int x = 0b11000;
%https://gcc.gnu.org/gcc-4.3/changes.html

Примеры целочисленных констант\footnote{
Используемые сокращения в заголовке таблицы:

\textbf{bin} %TODO

\textbf{oct} от octal -- восьмеричная система исчисления.

\textbf{dec} от decimal -- десятичная система исчисления.

\textbf{hex} от hexadecimal -- шестнадцатеричная система исчисления.
}:

\begin{small}
\begin{center}
\begin{tabular}{|r|r|r|r|}
\hline
$\textbf{bin}_{2}$ & $\textbf{oct}_{8}$ & $\textbf{dec}_{10}$ & $\textbf{hex}_{16}$\\
\hline
\texttt{0b0} & \texttt{0} & \texttt{0} & \texttt{0x0}\\
\texttt{0b111} & \texttt{07} & \texttt{7} & \texttt{0x7}\\
\texttt{0b1000} & \texttt{010} & \texttt{8} & \texttt{0x8}\\
\texttt{0b1010} & \texttt{012} & \texttt{10} & \texttt{0xA}\\
\texttt{0b1010101011111111} & \texttt{0125377} & \texttt{43775} & \texttt{0xAAFF}\\
\hline
\end{tabular}
\end{center}
\end{small}

Частой ошибкой является добавление <<ничего не значащего>> нуля в начало числа. Так следующая программа выведет <<неожиданный>> результат, хотя программист хотел вывести на экран число 45 в десятичной системе исчисления и случайной добавил ноль к начало числа:
\begin{lstlisting}
printf("%d\n", 045);
\end{lstlisting}
\begin{verbatim}
37
\end{verbatim} 

Хотя программа выдала верный результат ($045_{8} = 37_{10}$).

При попытке неправильного объявления целочисленной константы, компилятор выдаст ошибки:
\begin{lstlisting}
printf("%d", 049);  //error
printf("%d", 0x4W); //error
\end{lstlisting}
\begin{small}
\begin{verbatim}
1:14: error: invalid digit "9" in octal constant
2:14: error: invalid suffix "W" on integer constant
\end{verbatim} 
\end{small}

%TODO printf



















\subsection[Числа с плавающей запятой]{Числа с плавающей запятой \hfill\includegraphics[height=20pt]{images-gen/const-float.png}}

Действительные числа записываются как математике, только вместо разделителя целой и дробной части используется символ точки. Примеры таких констант:

\texttt{0.0 1.5 0.123455 127.0 4242.49}

Попытка использования запятой в качестве разделителя в числе приведёт к ошибке:
\begin{lstlisting}
float a = 0,1; //error
\end{lstlisting}
\begin{small}
\begin{verbatim}
1:13: error: expected identifier or ‘(’ before numeric 
  constant
\end{verbatim} 
\end{small}

Также язык Си поддерживает экспоненциальную запись\footnote{также известен как научный формат записи числа} действительных чисел, которая может быть удобна для записи слишком больших или слишком маленьких чисел:
\begin{lstlisting}
float x, y;
x = 5e2;  //$\color{red}x = 5 \times 10^2 = 5 \times 100 = 500.0$
y = 5e-2; //$\color{red}y = 5 \times 10^{-2} = 5 \times 0.01 = 0.05$
\end{lstlisting}
























\subsection[Символьные константы]{Символьные константы \hfill\includegraphics[height=20pt]{images-gen/const-char.png}}

Символьными константами являются чаще всего одиночные символы, заключённые в одинарные кавычки. Примеры таких констант: 

\texttt{'z' 'v' 'W'}

По умолчанию символьная константа имеет тип данных \texttt{int} (подробнее о типе данных %TODO int)

Некоторые символы не могут быть записаны одним символом. Для записи таких символов используются \textit{экранирующие последовательности}. Символ обратного слеша \texttt{\textbackslash} используется для \textit{экранирования} символа идущего за ним:

\begin{small}
\begin{center}
\begin{tabular}{|l|l|}
\hline
\textbf{Символ} & \textbf{Описание}\\
\hline
\texttt{\textbackslash{}0} & ноль-символ\\
\hline
\texttt{\textbackslash{}b} & backspace\\
\hline
\texttt{\textbackslash{}n} & новая строка\\
\hline
\texttt{\textbackslash{}r} & возврат каретки\\
\hline
\texttt{\textbackslash{}t} & горизонтальная табуляция\\
\hline
\texttt{\textbackslash{}?} & вопросительный знак\\
\hline
\texttt{\textbackslash{}'} & одинарная кавычка\\
\hline
\texttt{\textbackslash{}"} & двойная кавычка\\
\hline
\texttt{\textbackslash{}xh}, \texttt{\textbackslash{}xhh}, \texttt{\textbackslash{}xhhh}... & код символа в шестнадцатеричном виде\\
\hline
\end{tabular}
\end{center}
\end{small}

В памяти символы, в том числе и символьные константы хранятся в виде чисел из кодовой таблицы ASCII. Изначально (1963 год) кодировка ASCII была разработана для кодирования символов, коды которых помещались в 7 бит (128 символов; $2^7$). В дальнейшем кодировка была расширена до 8-ми бит (один байт) и в ней стало возможно хранение 256 символов ($2^8$). Восьмибитная кодировка получила название <<расширенной ASCII>>.

Значения кодов печатных символов всегда можно узнать в справочной таблице либо запустив следующий код:
\begin{lstlisting}
for (int i=32; i<128; i++) {
    printf("'%c' = %d\n", i, i);
}
\end{lstlisting}
Часть вывода программы:
\begin{verbatim}
'5' = 53
'6' = 54
'7' = 55
'8' = 56
'9' = 57
':' = 58
';' = 59
'<' = 60
'=' = 61
'>' = 62
'?' = 63
'@' = 64
'A' = 65
'B' = 66
'C' = 67
\end{verbatim} 

В диапазоне 0--31 находятся непечатные и управляющие символы (такие как перевод строки и возврат каретки), в диапазоне выше 128 -- дополнительные символы, включая национальные (кириллицу или другие).

\begin{lstlisting}
char ch1 = 'K';
char ch2 = '\x4b';  //$\color{red}4b_{16}=75_{10}$
char ch3 = 75;
printf("%c %c %c\n", ch1, ch2, ch3);
\end{lstlisting}
\begin{verbatim}
K K K
\end{verbatim} 






\subsection*{Замечание о кириллице и кодировке //todo}
\addcontentsline{toc}{subsection}{Замечание о кириллице и кодировке}

char ch = 'ж';

1:15: warning: multi-character character constant [-Wmultichar]

1:15: warning: overflow in implicit constant conversion [-Woverflow]


Произойдёт переполнение при неявном приведении типов  53430

+кирилица и корировка 866

соврадающая с кодировкой консоли.





















\subsection[Строковые константы]{Строковые константы \hfill\includegraphics[height=20pt]{images-gen/const-string.png}}

Строковые константы состоят из последовательности символов длиной ноль или более. В языке Си строки являются массивами символов и обязательно завершаются нуль-символом \texttt{\textbackslash{}0}. Размер строки нигде не хранится, и нуль-символ позволяет узнать, где заканчивается строка.

Строковые константы заключаются в двойные кавычки. Примеры строковых констант:

\texttt{"{}"}

\texttt{"tutti frutti ice cream"}

\texttt{"Hello World"}

Двойные кавычки внутри строки нужно экранировать:
\begin{lstlisting}
puts("Hello \"World\"");
\end{lstlisting}
Результат выполнения программы:
\begin{verbatim}
Hello "World"
\end{verbatim} 

Для строковых констант, написанных рядом без разделителя, происходит \textit{конкатенация} -- операция склеивания объектов линейной структуры. Эта особенность позволяет писать длинный текст на нескольких строках, а также полезна при написании макросов препроцессора.

\begin{lstlisting}
puts("Today's special is a pastrami sandwich "
     "on rye bread with a potato knish "
     "and a cherry soda.");
\end{lstlisting}

\begin{lstlisting}
"ice" "cream"  // <=> "icecream"
"ice " "cream" // <=> "ice cream"
"ice" " cream" // <=> "ice cream"
\end{lstlisting}

Строки могут содержать управляющие символы, например, символ новой строки:
\begin{lstlisting}
puts("hello\nworld");
\end{lstlisting}
Результат выполнения программы:
\begin{verbatim}
hello
world
\end{verbatim} 


















\section{Операторы}

\textit{Оператор} -- специальный токен, выполняющий операцию над выражениями. Примеры операторов: сложение \texttt{+}, вычитание \texttt{-}, умножение \texttt{*}.

По числу выражений разделяются на:
\begin{itemize}
\item \textit{унарные операторы} -- одно. Например, унарный минус для целочисленной константы: \texttt{-42};
\item \textit{бинарные операторы} -- два. Например, сложение двух чисел: \texttt{1 + 2};
\item \textit{тернарные операторы} -- три. Тернарный условный оператор (подробнее в %TODO)
\end{itemize}

Все операторы языка Си описаны в <<\nameref{ch:operators}>> (страница~\pageref{ch:operators}). %TODO search 'ref{'








\section{Разделители}

Разделители разделяют токены. Полный список разделителей в языке Си:

\texttt{( ) [ ] \{ \} ; , . :}

Пробельные символы тоже являются разделителями, но не являются токенами. Под пробельными символами понимается символ пробела, перевода строки, вертикальной и горизонтальной табуляции.

Пробельные символы игнорируются компилятором (за исключением пробельных символов внутри символьных и строковых констант), поэтому их использование является не обязательным для успешной компиляции.

Следующие две программы абсолютно одинаковые с точки зрения компилятора:


\lstinputlisting{examples/2-hello.c}
\begin{lstlisting}
#include <stdio.h> int main(){puts("Hello World!");return 0;}
\end{lstlisting}

Конечно же не стоит писать, как приведено во втором варианте, это снижает читабельность текста.

Приведённый выше код будет разобран на следующие токены (содержимое файла \texttt{stdio.h} не показано):

\begin{small}
\begin{center}
\begin{tabular}{|l|l|}
\hline
\textbf{Токен} & \textbf{Описание}\\
\hline
\texttt{int} & ключевое слово\\
\texttt{main} & идентификатор\\
\texttt{(} & разделитель\\
\texttt{)} & разделитель\\
\texttt{\{} & разделитель\\
\texttt{puts} & идентификатор\\
\texttt{(} & разделитель\\
\texttt{"Hello World!"} & строковая константа\\
\texttt{)} & разделитель\\
\texttt{;} & разделитель\\
\texttt{return} & ключевое слово\\
\texttt{0} & целочисленная константа\\
\texttt{;} & разделитель\\
\texttt{\}} & разделитель\\
\texttt{EOF}\footnote{аббревиатура end of file} & конец файла\\
\hline
\end{tabular}
\end{center}
\end{small}







\section*{code style //todo}
\addcontentsline{toc}{section}{code style //todo}
%В этой главе рассмотрены исходные кода пустой программы, которая ничего не делает, и <<Hello World>>.

%\lstinputlisting[caption=Пустая программа void.c]{examples/1-void.c}

%Выше приведён исходный код пустой программы. На первый взгляд она бесполезна, однако даже она достойна внимания по нескольким причинам:
%\begin{enumerate}
%\item код синтаксически корректен и компилируется;
%\item программа успешно запускается и выгружается из памяти;
%\item проста для понимания.
%\end{enumerate}


\lstinputlisting[showspaces=true]{examples/2-hello.c}

Обратите внимание на пробельные отступы внутри блока (между символами \texttt{\{} и \texttt{\}}). Пробелы нужны для визуального оформления кода и индикации уровня вложенности. Рекомендуется придерживаться единого стиля написания кода и использовать четыре пробела или пользоваться клавишей \texttt{Tab}.



% Компилятор парсит исходный код и на основе грамматики языка <<видит>> написанное. Так он обнаружит определение функции с именем \texttt{main} без входных аргументов (ничего нет между круглыми скобками) и возвращающую целочисленное значение. Далее он перейдёт внутрь тела функции и обработает все операторы внутри.

%Итогом компиляции станет выполняемый файл, содержащий внутри одну функцию \texttt{main} -- точку входа в программу, наличие которой обязательно требуется.

%Подробнее о функции \texttt{main} рассказывается на странице~\pageref{ch:main} в главе~\ref{ch:main}~<<\nameref{ch:main}>>.

%Найдите рядом лежащий рядом с исходным файл и запустите \texttt{void.exe} (в Windows) или просто \texttt{void} (в Unix).

%Программа должна успешно выполниться и завершиться без ошибок.

%Дополнительно можно проверить в консоли код возврата (число, которое вернула функция \texttt{main} с помощью оператора \texttt{return}):

%\begin{lstlisting}[language=bash]
%void.exe && echo ok
%\end{lstlisting}










\vfill
\subsection*{Контрольные вопросы}
\begin{enumerate}
\item 
\end{enumerate}
\end{document}
